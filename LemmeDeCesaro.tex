%% LyX 2.2.1 created this file.  For more info, see http://www.lyx.org/.
%% Do not edit unless you really know what you are doing.
\documentclass[french,english,12pt]{paper}
\usepackage[T1]{fontenc}
\usepackage[latin9]{inputenc}
\usepackage[landscape,a4paper]{geometry}
\geometry{verbose,tmargin=40mm,bmargin=0mm,lmargin=30mm,rmargin=30mm}
\pagestyle{empty}
\setlength{\parskip}{\bigskipamount}
\setlength{\parindent}{0pt}
\usepackage{fancybox}
\usepackage{calc}
\usepackage{amsmath}
\usepackage{amssymb}

\makeatletter
%%%%%%%%%%%%%%%%%%%%%%%%%%%%%% User specified LaTeX commands.
\usepackage{tikz}
\usetikzlibrary{calc}
\usetikzlibrary{shapes}
\usepackage{mathrsfs}
\usepackage{mathabx}
\usepackage{txfonts}
\usepackage{pxfonts}
\usepackage{titling}
\usepackage{array}
%\usepackage{yhmath}

\newdimen\un \un=1mm
\def\bordure{
\begin{tikzpicture}[overlay,remember picture]
\node(Triskell) at ($(current page.north west)+(21*\un,-20*\un)$){};
\def\p{.75}
\def\ang{45}
\def\alp{160.2}
\def\bet{72.42}
\def\gam{-13.2}
\draw [very thick, line width = 8pt, color = red!25!blue!33.333!green!50]
($(current page.north west)+(20*\un,-20*\un)$)
-- ($(current page.north east)+(-24*\un,-20*\un)$)
node [draw, ellipse, fill, text=white, pos=.5] {\thetitle}
arc (90 : 0 : 4*\un)
-- ($(current page.south east)+(-20*\un,24*\un)$)
node[text=white,pos=.5 , rotate=90]
{\tiny Ce document est sous licence GNU FDL,
 il est librement modifiable et distribuable.
 Sources et licence compl�tes disponible sur le site.
 Copyright 2015, Jean-Christophe Jameux}
arc (0 : -90 : 4*\un)
-- ($(current page.south west)+(24*\un,20*\un)$)
node[draw, ellipse, fill, text=white, pos=.15] {\bf Echologie.org}
arc (-90 : -180 : 4*\un)
-- ($(current page.north west)+(20*\un,-20*\un)$);
\draw [fill = white, color = red!25!blue!33.333!green!50]
(Triskell) + (1.2*\un,-6*\un) circle (15*\un);
\draw [fill = white, color = white] (Triskell) circle (2*\un);
\draw [fill = white, color = white]
(Triskell) + ({120*(1+\p)} : 3*\un)
arc ({120*(1+\p)} : 120 : 3*\un)
arc (180+\ang : 180-\ang :3*\un)
arc (\alp-\ang : \alp+\ang+24.8 : 5*\un);
\draw [fill = white, color = white]
(Triskell) + (120*\p : 3*\un)
arc (120*\p : 0 : 3*\un)
arc (90+\ang : 90-\ang : 6*\un)
arc (\bet-\ang : \bet+\ang+5.9 : 8*\un);
\draw [fill = white, color = white]
(Triskell) + ({120*(2+\p)} : 3*\un)
arc ({120*(2+\p)} : 240 : 3*\un)
arc (\ang : -\ang : 12*\un)
arc (\gam-\ang : \gam+\ang+.85 : 13*\un);
\end{tikzpicture}}

\makeatother

\usepackage{babel}
\makeatletter
\addto\extrasfrench{%
   \providecommand{\og}{\leavevmode\flqq~}%
   \providecommand{\fg}{\ifdim\lastskip>\z@\unskip\fi~\frqq}%
}

\makeatother
\begin{document}
\selectlanguage{french}%
\title{\LARGE\bf Le lemme de Ces�ro}

\textbf{\large{}Soit $(u_{n})_{n\in\mathbb{N}^{*}}$ une suite r�elle
convergeant vers 0. On se propose de d�montrer que la suite ${\displaystyle \left(\frac{\sum_{k=1}^{n}u_{k}}{n}\right)_{n\in\mathbb{N}^{*}}}$
converge �galement vers 0. Il s'agit donc de d�montrer $\forall\varepsilon>0,\exists N\in\mathbb{N}^{*},\forall n\geq N,\left|\frac{\sum_{k=1}^{n}u_{k}}{n}\right|<\varepsilon$}{\large \par}

\bigskip{}

\textbf{\large{}}%
{\fboxsep 3mm\Ovalbox{\begin{minipage}[t]{1\columnwidth - 2\fboxsep - 1.6pt}%
\textbf{\large{}Soit $\varepsilon>0$, puisque ${\displaystyle \lim_{n\longrightarrow+\infty}u_{n}=0}$,
on sait qu'il existe $N_{1}\in\mathbb{N}^{*}$ v�rifiant $\forall n\geq N_{1},|u_{n}|<\frac{\varepsilon}{2}$}{\large \par}

\textbf{\large{}Montrons alors $\forall n\geq N_{1},\left|\frac{\sum_{k=N_{1}}^{n}u_{k}}{n}\right|<\frac{\varepsilon}{2}$}{\large \par}

\bigskip{}

\textbf{\large{}}%
{\fboxsep 3mm\Ovalbox{\begin{minipage}[t]{1\columnwidth - 2\fboxsep - 1.6pt}%
\textbf{\large{}Soit donc $n\geq N_{1}$, on peut d�s lors affirmer
:
\[
{\displaystyle \left|\frac{\sum_{k=N_{1}}^{n}u_{k}}{n}\right|\leq\frac{\sum_{k=N_{1}}^{n}|u_{k}|}{n}<\frac{(n-N_{1}+1)\cdot\frac{\varepsilon}{2}}{n}\leq\frac{n\cdot\frac{\varepsilon}{2}}{n}=\frac{\varepsilon}{2}}
\]
}%
\end{minipage}}}{\large \par}

\bigskip{}
\textbf{\large{}De m�me, puisque ${\displaystyle \lim_{n\longrightarrow+\infty}\frac{\sum_{k=1}^{N_{1}-1}u_{k}}{n}=0}$,
on sait qu'il existe $N_{2}\in\mathbb{N}^{*}$ v�rifiant $\forall n\geq N_{2},\left|\frac{\sum_{k=1}^{N_{1}-1}u_{k}}{n}\right|<\frac{\varepsilon}{2}$.}{\large \par}

\textbf{\large{}En posant $N=\max(N_{1},N_{2})$, on d�duit de ce
qui pr�c�de que pour tout $n\geq N$ on a la majoration} :

\textbf{\large{}
\[
{\displaystyle \left|\frac{\sum_{k=1}^{n}u_{k}}{n}\right|=\left|\frac{\sum_{k=1}^{N_{1}-1}u_{k}}{n}+\frac{\sum_{k=N_{1}}^{n}u_{k}}{n}\right|\leq\left|\frac{\sum_{k=1}^{N_{1}-1}u_{k}}{n}\right|+\left|\frac{\sum_{k=N_{1}}^{n}u_{k}}{n}\right|<\frac{\varepsilon}{2}+\frac{\varepsilon}{2}=\varepsilon}
\]
}{\large \par}%
\end{minipage}}}{\large \par}

\bordure\selectlanguage{english}%

\end{document}
