%% LyX 2.2.1 created this file.  For more info, see http://www.lyx.org/.
%% Do not edit unless you really know what you are doing.
\documentclass[french,12pt]{paper}
\usepackage[T1]{fontenc}
\usepackage[utf8]{inputenc}
\usepackage[landscape,a4paper]{geometry}
\geometry{verbose,tmargin=30mm,bmargin=20mm,lmargin=35mm,rmargin=35mm}
\pagestyle{empty}
\setlength{\parskip}{\bigskipamount}
\setlength{\parindent}{0pt}
\usepackage{amsmath}
\usepackage{amssymb}

\makeatletter
%%%%%%%%%%%%%%%%%%%%%%%%%%%%%% User specified LaTeX commands.
\usepackage{tikz}
\usetikzlibrary{calc}
\usetikzlibrary{shapes}
\usepackage{mathrsfs}
\usepackage{mathabx}
\usepackage{txfonts}
\usepackage{pxfonts}
\usepackage{titling}
\usepackage{array}

\newdimen\un \un=1mm
\def\bordure{
\begin{tikzpicture}[overlay,remember picture]
\node(Triskell) at ($(current page.north west)+(21*\un,-20*\un)$){};
\def\p{.75}
\def\ang{45}
\def\alp{160.2}
\def\bet{72.42}
\def\gam{-13.2}
\draw [very thick, line width = 8pt, color = red!25!blue!33.333!green!50]
($(current page.north west)+(20*\un,-20*\un)$)
-- ($(current page.north east)+(-24*\un,-20*\un)$)
node [draw, ellipse, fill, text=white, pos=.5] {\thetitle}
arc (90 : 0 : 4*\un)
-- ($(current page.south east)+(-20*\un,24*\un)$)
node[text=white,pos=.5 , rotate=90]
{\tiny Ce document est sous licence GNU FDL,
 il est librement modifiable et distribuable.
 Sources et licence complètes disponible sur le site.
 Copyright 2016, Jean-Christophe Jameux}
arc (0 : -90 : 4*\un)
-- ($(current page.south west)+(24*\un,20*\un)$)
node[draw, ellipse, fill, text=white, pos=.15] {\bf Echologie.org}
arc (-90 : -180 : 4*\un)
-- ($(current page.north west)+(20*\un,-20*\un)$);
\draw [fill = white, color = red!25!blue!33.333!green!50]
(Triskell) + (1.2*\un,-6*\un) circle (15*\un);
\draw [fill = white, color = white] (Triskell) circle (2*\un);
\draw [fill = white, color = white]
(Triskell) + ({120*(1+\p)} : 3*\un)
arc ({120*(1+\p)} : 120 : 3*\un)
arc (180+\ang : 180-\ang :3*\un)
arc (\alp-\ang : \alp+\ang+24.8 : 5*\un);
\draw [fill = white, color = white]
(Triskell) + (120*\p : 3*\un)
arc (120*\p : 0 : 3*\un)
arc (90+\ang : 90-\ang : 6*\un)
arc (\bet-\ang : \bet+\ang+5.9 : 8*\un);
\draw [fill = white, color = white]
(Triskell) + ({120*(2+\p)} : 3*\un)
arc ({120*(2+\p)} : 240 : 3*\un)
arc (\ang : -\ang : 12*\un)
arc (\gam-\ang : \gam+\ang+.85 : 13*\un);
\end{tikzpicture}}

\makeatother

\usepackage{babel}
\makeatletter
\addto\extrasfrench{%
   \providecommand{\og}{\leavevmode\flqq~}%
   \providecommand{\fg}{\ifdim\lastskip>\z@\unskip\fi~\frqq}%
}

\makeatother
\begin{document}
\title{Mise en équations}

\textbf{\large{}\hphantom{\textbf{\large{}D}}Dans la boîte à outil
du mathématicien, la mise en équation est l'arme la plus destructrice...
De difficultés ! Donnons-en un exemple : }{\large \par}
\begin{center}
\textbf{\large{}}%
\begin{minipage}[t]{0.8\columnwidth}%
\begin{center}
\textbf{\large{}Un nombre entier a pour reste 35 dans la division
euclidienne par 69, et dans la division par 75, il a même quotient
et pour reste 17. Quel est ce nombre ?}
\par\end{center}%
\end{minipage}
\par\end{center}{\large \par}

\textbf{\large{}On peut tout à fait trouver la solution de ce problème
en tatonant, mais rien ne garantit qu'on aboutira, et pire, face à
un autre problème, on repart de zéro. La clef de la mise en équation,
c'est de nomer les éléments du problème, afin de pouvoir formuler
le problème comme si l'on connaissait la solution. Dans notre exemple,
on peut appeler $n$ le nombre entier recherché, et $q$ le quotient
commun. La résolution du problème arrive très vite après la mise en
équation :
\begin{align*}
\begin{cases}
n=69q+35\\
n=75q+17
\end{cases} & \Longrightarrow0=-6q+18 & \textrm{En soustrayant les deux lignes}\\
 & \Longrightarrow q=3 & \textrm{On trouve ainsi le quotient commun}\\
 & \Longrightarrow n=69\times3+35 & \textrm{On substitue dans la première équation}\\
 & \Longrightarrow n=242 & \textrm{(ou la deuxième : \ensuremath{n=75\times3+17=242})}
\end{align*}
En faisant ainsi, on ne démontre pas que 242 est une solution de notre
problème, puisque nous avons des implications et non des équivalences.
Ce que nous avons montré, c'est que seul 242 peut être solution, mais
c'est un jeu d'enfant de vérifier que ce nombre répond bien aux conditions
du problème. On a ainsi fait mieux que trouver une solution du problème,
on a prouvé qu'il n'y en avait pas d'autres, et on a surtout, en prenant
un peu de recul, mis en pratique une technique de résolution de problème
très puissante et très générale.}{\large \par}

\bordure
\end{document}
