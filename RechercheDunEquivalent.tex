%% LyX 2.2.1 created this file.  For more info, see http://www.lyx.org/.
%% Do not edit unless you really know what you are doing.
\documentclass[french,english,12pt]{paper}
\usepackage[T1]{fontenc}
\usepackage[latin9]{inputenc}
\usepackage[landscape,a4paper]{geometry}
\geometry{verbose,tmargin=27mm,bmargin=0mm,lmargin=28mm,rmargin=28mm}
\pagestyle{empty}
\setlength{\parskip}{\bigskipamount}
\setlength{\parindent}{0pt}
\usepackage{amsmath}
\usepackage{amssymb}
\usepackage{stackrel}

\makeatletter
%%%%%%%%%%%%%%%%%%%%%%%%%%%%%% User specified LaTeX commands.
\usepackage{tikz}
\usetikzlibrary{calc}
\usetikzlibrary{shapes}
\usepackage{mathrsfs}
\usepackage{mathabx}
\usepackage{txfonts}
\usepackage{pxfonts}
\usepackage{titling}
\usepackage{array}

\newdimen\un \un=1mm
\def\bordure{
\begin{tikzpicture}[overlay,remember picture]
\node(Triskell) at ($(current page.north west)+(21*\un,-20*\un)$){};
\def\p{.75}
\def\ang{45}
\def\alp{160.2}
\def\bet{72.42}
\def\gam{-13.2}
\draw[very thick, line width = 8pt, color=red!25!blue!33.333!green!50]
($(current page.north west)+(20*\un,-20*\un)$)
-- ($(current page.north east)+(-24*\un,-20*\un)$)
node [draw, ellipse, fill, text=white, pos=.5] {\thetitle}
arc (90 : 0 : 4*\un)
-- ($(current page.south east)+(-20*\un,24*\un)$)
node[text=white,pos=.5 , rotate=90]
{\tiny Ce document est sous licence GNU FDL,
 il est librement modifiable et distribuable.
 Sources et licence compl�tes disponible sur le site.
 Copyright 2016, Jean-Christophe Jameux}
arc (0 : -90 : 4*\un)
-- ($(current page.south west)+(24*\un,20*\un)$)
node[draw, ellipse, fill, text=white, pos=.15] {Echologie.org}
arc (-90 : -180 : 4*\un)
-- ($(current page.north west)+(20*\un,-20*\un)$);
\draw[fill=white,color=red!25!blue!33.333!green!50]
(Triskell) + (1.2*\un,-6*\un) circle (15*\un);
\draw[fill=white,color=white] (Triskell) circle (2*\un);
\draw[fill=white,color=white] (Triskell) + ({120*(1+\p)} : 3*\un)
arc ({120*(1+\p)} : 120 : 3*\un)
arc (180+\ang : 180-\ang :3*\un)
arc (\alp-\ang : \alp+\ang+24.8 : 5*\un);
\draw[fill=white,color=white] (Triskell) + (120*\p : 3*\un)
arc (120*\p : 0 : 3*\un)
arc (90+\ang : 90-\ang : 6*\un)
arc (\bet-\ang : \bet+\ang+5.9 : 8*\un);
\draw[fill=white,color=white] (Triskell) + ({120*(2+\p)} : 3*\un)
arc ({120*(2+\p)} : 240 : 3*\un)
arc (\ang : -\ang : 12*\un)
arc (\gam-\ang : \gam+\ang+.85 : 13*\un);
\end{tikzpicture}}

\makeatother

\usepackage{babel}
\makeatletter
\addto\extrasfrench{%
   \providecommand{\og}{\leavevmode\flqq~}%
   \providecommand{\fg}{\ifdim\lastskip>\z@\unskip\fi~\frqq}%
}

\makeatother
\begin{document}
\selectlanguage{french}%
\title{\large\bf Recherche d'un �quivalent}

\selectlanguage{english}%
\begin{tikzpicture}[scale=1,overlay,xshift=40,yshift=-400]
	\shorthandoff{:}
	\draw[very thick,->,>=latex] (-1,0) -- (5,0);
	\draw[very thick,->,>=latex] (0,-1) -- (0,10);
	\draw[very thick, color=red!25!blue!33.333!green!50] plot[smooth,domain=-1:.836] (\x,{.3*(3*(2-2^(.5))/(2*(\x-1)^2)});
	\draw[very thick, color=red!25!blue!33.333!green!50] plot[smooth,domain=1.164:5] (\x,{.3*(3*(2-2^(.5))/(2*(\x-1)^2)});
	\draw[very thick, color=purple, opacity=.5] plot[smooth,domain=-1:.831] (\x,{.3*((2*\x+2)^(.5)-(\x^2+1)^(.5))/((8*\x^2-16*\x+16)^(1/3)-2)});
	\draw[very thick, color=purple, opacity=.5] plot[smooth,domain=1.16:5] (\x,{.3*((2*\x+2)^(.5)-(\x^2+1)^(.5))/((8*\x^2-16*\x+16)^(1/3)-2)});
\node[color=purple] at (3.5,7) {$\mathscr C_f:y=\frac{\sqrt{2x+2}-\sqrt{x^{2}+1}}{\sqrt[3]{8x^{2}-16x+16}-2}$};
\node[color=red!25!blue!33.333!green!50] at (3,4) {$\mathscr C_g:y=\frac{3\left(2-\sqrt{2}\right)}{2\left(x-1\right)^{2}}$};
\end{tikzpicture}\textbf{\small{}\phantom{\textbf{\small{}Decale.}}On cherche � d�terminer
un �quivalent simple de la fonction $f$ au voisinage de $1$ gr�ce
� un changement de variable. Apr�s \phantom{\textbf{\small{}Decal.}}quelques
lignes plus ou moins laborieuses de calculs pas sp�cialement intelligents,
on obtiendra l'�quivalent $g$ qui, � une translation \phantom{\textbf{\small{}Dec}}et
� une multiplication par une constante pr�s, n'est autre que la fonction
$x\mapsto\frac{1}{x^{2}}$. Le graphique ci-dessous permet de voir
la qualit� de l'approximation !\vspace{-15mm}
}{\small \par}

\textbf{\small{}
\begin{align*}
\textrm{Pour }x\textrm{ voisin de }1\textrm{, on a }\frac{\sqrt{2x+2}-\sqrt{x^{2}+1}}{\sqrt[3]{8x^{2}-16x+16}-2} & =\frac{\sqrt{2\left(1+h\right)+2}-\sqrt{\left(1+h\right)^{2}+1}}{\sqrt[3]{8\left(1+h\right)^{2}-16\left(1+h\right)+16}-2} & \textrm{En posant }x=1+h\\
 & =\frac{\sqrt{4+2h}-\sqrt{2+2h+h^{2}}}{\sqrt[3]{8\left(1+h^{2}\right)}-2} & \textrm{On cherche � utiliser la formule :}\\
 & =\frac{2\cdot\sqrt{1+\frac{h}{2}}-\sqrt{2}\cdot\sqrt{1+\left(h+\frac{h^{2}}{2}\right)}}{2\cdot\sqrt[3]{1+h^{2}}-2} & \left(1+x\right)^{\alpha}\stackrel[x\to0]{}{=}1+\alpha x+o\left(x\right)\\
 & =\frac{2\left(1+\frac{1}{4}h+o\left(h\right)\right)-\sqrt{2}\left(1+\frac{1}{2}h+o\left(h\right)\right)}{2\left(1+\frac{1}{3}h^{2}+o\left(h^{2}\right)\right)-2} & \textrm{Car }o\left(h+\frac{h^{2}}{2}\right)\stackrel[h\to0]{}{=}o\left(h\right)\\
 & =\frac{2-\sqrt{2}+\frac{1-\sqrt{2}}{2}h+o\left(h\right)}{\frac{2}{3}h^{2}+o\left(h^{2}\right)} & \textrm{et }\frac{h^{2}}{2}\stackrel[h\to0]{}{=}o\left(h\right)\\
 & =\frac{3}{2h^{2}}\cdot\frac{2-\sqrt{2}+\frac{1-\sqrt{2}}{2}h+o\left(h\right)}{1+o\left(1\right)} & \textrm{On cherche � utiliser la formule :}\\
 & =\frac{3}{2h^{2}}\cdot\left(2-\sqrt{2}+\frac{1-\sqrt{2}}{2}h+o\left(h\right)\right)\left(1+o\left(1\right)\right) & \frac{1}{1+x}\stackrel[x\to0]{}{=}1-x+o\left(x\right)\\
 & =\frac{3}{2h^{2}}\cdot\left(2-\sqrt{2}+o\left(1\right)\right) & \textrm{(cas particulier de la formule pr�c�dente)}\\
 & =\frac{3\left(2-\sqrt{2}\right)}{2h^{2}}+o\left(\frac{1}{h^{2}}\right) & \textrm{En d�veloppant}\\
 & =\frac{3\left(2-\sqrt{2}\right)}{2\left(x-1\right)^{2}}+o\left(\frac{1}{\left(x-1\right)^{2}}\right) & \textrm{En repassant � la variable }x\\
 & \sim\frac{3\left(2-\sqrt{2}\right)}{2\left(x-1\right)^{2}} & \textrm{Terme d'erreur n�gligeable devant la fonction}
\end{align*}
}{\small \par}

\selectlanguage{french}%
\bordure\selectlanguage{english}%

\end{document}
