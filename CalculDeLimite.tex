%% LyX 2.2.1 created this file.  For more info, see http://www.lyx.org/.
%% Do not edit unless you really know what you are doing.
\documentclass[french,english,12pt]{paper}
\usepackage[T1]{fontenc}
\usepackage[latin9]{inputenc}
\usepackage[a4paper]{geometry}
\geometry{verbose,tmargin=34mm,bmargin=0mm,lmargin=30mm,rmargin=30mm}
\pagestyle{empty}
\setlength{\parskip}{\bigskipamount}
\setlength{\parindent}{0pt}
\usepackage{amsmath}
\usepackage{amssymb}

\makeatletter
%%%%%%%%%%%%%%%%%%%%%%%%%%%%%% User specified LaTeX commands.
\usepackage{tikz}
\usetikzlibrary{calc}
\usetikzlibrary{shapes}
\usepackage{mathrsfs}
\usepackage{mathabx}
\usepackage{txfonts}
\usepackage{pxfonts}
\usepackage{titling}
\usepackage{array}
%\usepackage{yhmath}

\newdimen\un \un=1mm
\def\bordure{
\begin{tikzpicture}[overlay,remember picture]
\node(Triskell) at ($(current page.north west)+(21*\un,-20*\un)$){};
\def\p{.75}
\def\ang{45}
\def\alp{160.2}
\def\bet{72.42}
\def\gam{-13.2}
\draw [very thick, line width = 8pt, color = red!25!blue!33.333!green!50]
($(current page.north west)+(20*\un,-20*\un)$)
-- ($(current page.north east)+(-24*\un,-20*\un)$)
node [draw, ellipse, fill, text=white, pos=.5] {\thetitle}
arc (90 : 0 : 4*\un)
-- ($(current page.south east)+(-20*\un,24*\un)$)
node[text=white,pos=.65 , rotate=90]
{\tiny Ce document est sous licence GNU FDL,
 il est librement modifiable et distribuable.
 Licence et sources compl�tes disponibles sur le site.
 Copyright 2016, Jean-Christophe Jameux}
arc (0 : -90 : 4*\un)
-- ($(current page.south west)+(24*\un,20*\un)$)
node[draw, ellipse, fill, text=white, pos=.2] {\bf\large Echologie.org}
arc (-90 : -180 : 4*\un)
-- ($(current page.north west)+(20*\un,-20*\un)$);
\draw [fill = white, color = red!25!blue!33.333!green!50]
(Triskell) + (1.2*\un,-6*\un) circle (15*\un);
\draw [fill = white, color = white] (Triskell) circle (2*\un);
\draw [fill = white, color = white]
(Triskell) + ({120*(1+\p)} : 3*\un)
arc ({120*(1+\p)} : 120 : 3*\un)
arc (180+\ang : 180-\ang :3*\un)
arc (\alp-\ang : \alp+\ang+24.8 : 5*\un);
\draw [fill = white, color = white]
(Triskell) + (120*\p : 3*\un)
arc (120*\p : 0 : 3*\un)
arc (90+\ang : 90-\ang : 6*\un)
arc (\bet-\ang : \bet+\ang+5.9 : 8*\un);
\draw [fill = white, color = white]
(Triskell) + ({120*(2+\p)} : 3*\un)
arc ({120*(2+\p)} : 240 : 3*\un)
arc (\ang : -\ang : 12*\un)
arc (\gam-\ang : \gam+\ang+.85 : 13*\un);
\end{tikzpicture}}

\makeatother

\usepackage{babel}
\makeatletter
\addto\extrasfrench{%
   \providecommand{\og}{\leavevmode\flqq~}%
   \providecommand{\fg}{\ifdim\lastskip>\z@\unskip\fi~\frqq}%
}

\makeatother
\begin{document}
\selectlanguage{french}%
\title{\LARGE\bf Calcul de limite}

\selectlanguage{english}%
\textbf{\large{}\hphantom{Dec}Quand on souhaite r�aliser un calcul
de limite et que l'on est face � une forme ind�termin�e, il est judicieux
de commencer par faire les calculs ``� la louche'' afin d'avoir
une id�e de la limite. Il devient d�s lors plus simple de voir comment
lever l'ind�termination. Par exemple, quand $x$ est proche de $-\infty$
:}{\large \par}

\textbf{
\begin{align*}
\frac{\sqrt{x^{2}+1}-2x}{x^{2}} & \approx\frac{\sqrt{x^{2}}-2x}{x^{2}} & \textrm{Car \ensuremath{1} est n�gligeable devant \ensuremath{x^{2}}}\\
 & \approx\frac{-x-2x}{x^{2}} & \textrm{Car quand \ensuremath{x} est n�gatif, }\sqrt{x^{2}}=\left|x\right|=-x\\
 & \approx\frac{-3x}{x^{2}} & \textrm{On a toujours une forme ind�termin�e : \ensuremath{\frac{+\infty}{+\infty}}}\\
 & \approx\frac{-3}{x} & \textrm{(ce nombre est positif par la r�gle des signes)}\\
 & \approx0 & \textrm{(et m�me \ensuremath{0^{+}})}
\end{align*}
}

\textbf{\large{}Nos calculs � la louche nous incitent � faire appara�tre
un $-x$ dans l'�criture de $\sqrt{x^{2}+1}$. Voyons ce que cela
donne :}\textbf{
\begin{align*}
 & \phantom{=.}\lim_{x\to-\infty}\frac{\sqrt{x^{2}+1}-2x}{x^{2}}\\
 & =\lim_{x\to-\infty}\frac{\sqrt{x^{2}\cdot\left(1+\frac{1}{x^{2}}\right)}-2x}{x^{2}} & \textrm{On met le \ensuremath{x^{2}} en facteur sous la racine}\\
 & =\lim_{x\to-\infty}\frac{\sqrt{x^{2}}\cdot\sqrt{1+\frac{1}{x^{2}}}-2x}{x^{2}} & \textrm{(il domine le polyn�me \ensuremath{x^{2}+1} en \ensuremath{+\infty})}\\
 & =\lim_{x\to-\infty}\frac{\left|x\right|\cdot\sqrt{1+\frac{1}{x^{2}}}-2x}{x^{2}} & \textrm{afin de le s�parer du \ensuremath{1+\frac{1}{x^{2}}}}\\
 & =\lim_{x\to-\infty}\frac{-x\cdot\sqrt{1+\frac{1}{x^{2}}}-2x}{x^{2}} & \textrm{(qui tend vers \ensuremath{1} quand \ensuremath{x} tend vers \ensuremath{-\infty})}\\
 & =\lim_{x\to-\infty}\frac{x\cdot\left(-\sqrt{1+\frac{1}{x^{2}}}-2\right)}{x^{2}} & {\scriptscriptstyle \left(-\sqrt{1+\frac{1}{x^{2}}}-2\right)}\textrm{ joue le r�le du -3 du dessus}\\
 & =\lim_{x\to-\infty}\frac{\left(-\sqrt{1+\frac{1}{x^{2}}}-2\right)}{x} & \textrm{On simplifie les infinis "de m�me taille"}\\
 & =0 & \textrm{Car }\lim_{x\to-\infty}-\sqrt{1+\frac{1}{x^{2}}}-2=-3
\end{align*}
}

\selectlanguage{french}%
\bordure\selectlanguage{english}%

\end{document}
