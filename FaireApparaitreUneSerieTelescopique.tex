%% LyX 2.2.1 created this file.  For more info, see http://www.lyx.org/.
%% Do not edit unless you really know what you are doing.
\documentclass[french,english,12pt]{paper}
\usepackage[T1]{fontenc}
\usepackage[latin9]{inputenc}
\usepackage[landscape,a4paper]{geometry}
\geometry{verbose,tmargin=29mm,bmargin=0mm,lmargin=28mm,rmargin=28mm}
\pagestyle{empty}
\setlength{\parskip}{\bigskipamount}
\setlength{\parindent}{0pt}
\usepackage{amsmath}
\usepackage{amssymb}

\makeatletter
%%%%%%%%%%%%%%%%%%%%%%%%%%%%%% User specified LaTeX commands.
\usepackage{tikz}
\usetikzlibrary{calc}
\usetikzlibrary{shapes}
\usepackage{mathrsfs}
\usepackage{mathabx}
\usepackage{txfonts}
\usepackage{pxfonts}
\usepackage{titling}
\usepackage{array}
%\usepackage{yhmath}

\newdimen\un \un=1mm
\def\bordure{
\begin{tikzpicture}[overlay,remember picture]
\node(Triskell) at ($(current page.north west)+(21*\un,-20*\un)$){};
\def\p{.75}
\def\ang{45}
\def\alp{160.2}
\def\bet{72.42}
\def\gam{-13.2}
\draw [very thick, line width = 8pt, color = red!25!blue!33.333!green!50]
($(current page.north west)+(20*\un,-20*\un)$)
-- ($(current page.north east)+(-24*\un,-20*\un)$)
node [draw, ellipse, fill, text=white, pos=.5] {\thetitle}
arc (90 : 0 : 4*\un)
-- ($(current page.south east)+(-20*\un,24*\un)$)
node[text=white,pos=.5 , rotate=90]
{\tiny Ce document est sous licence GNU FDL,
 il est librement modifiable et distribuable.
 Sources et licence compl�tes disponible sur le site.
 Copyright 2015, Jean-Christophe Jameux}
arc (0 : -90 : 4*\un)
-- ($(current page.south west)+(24*\un,20*\un)$)
node[draw, ellipse, fill, text=white, pos=.15] {\bf Echologie.org}
arc (-90 : -180 : 4*\un)
-- ($(current page.north west)+(20*\un,-20*\un)$);
\draw [fill = white, color = red!25!blue!33.333!green!50]
(Triskell) + (1.2*\un,-6*\un) circle (15*\un);
\draw [fill = white, color = white] (Triskell) circle (2*\un);
\draw [fill = white, color = white]
(Triskell) + ({120*(1+\p)} : 3*\un)
arc ({120*(1+\p)} : 120 : 3*\un)
arc (180+\ang : 180-\ang :3*\un)
arc (\alp-\ang : \alp+\ang+24.8 : 5*\un);
\draw [fill = white, color = white]
(Triskell) + (120*\p : 3*\un)
arc (120*\p : 0 : 3*\un)
arc (90+\ang : 90-\ang : 6*\un)
arc (\bet-\ang : \bet+\ang+5.9 : 8*\un);
\draw [fill = white, color = white]
(Triskell) + ({120*(2+\p)} : 3*\un)
arc ({120*(2+\p)} : 240 : 3*\un)
arc (\ang : -\ang : 12*\un)
arc (\gam-\ang : \gam+\ang+.85 : 13*\un);
\end{tikzpicture}}

\makeatother

\usepackage{babel}
\makeatletter
\addto\extrasfrench{%
   \providecommand{\og}{\leavevmode\flqq~}%
   \providecommand{\fg}{\ifdim\lastskip>\z@\unskip\fi~\frqq}%
}

\makeatother
\begin{document}
\selectlanguage{french}%
\title{\large\bf Faire appara�tre une s�rie t�lescopique}

\selectlanguage{english}%
\textbf{\hphantom{Decale}Dans certaines situations, il est possible
de grandement simplifier le calcul d'une somme en mettant son terme
g�n�ral \hphantom{Deca.}sous une forme qui permet la simplification
de la presque totalit� des termes. C'est le principe du t�lescopage,
que l'on \hphantom{De}r�sume par la formule ${\displaystyle \sum_{k=0}^{n}\left(a_{k}-a_{k+1}\right)}=a_{\text{0}}-a_{n+1}$
o� $a$ d�signe une suite absolument quelconque. Le calcul ci-dessous
montre un exemple de la mise en �uvre de cette technique :}\textbf{\large{}
\[
\begin{aligned}\sum_{p=0}^{n}p\cdot p! & =\sum_{p=1}^{n}p\cdot p! & \text{car quand }p=0,\ p\cdot p!=0\\
 & =\sum_{p=1}^{n}(p+1-1)\cdot p!\\
 & =\sum_{p=1}^{n}\big((p+1)\cdot p!-p!\big)\\
 & =\sum_{p=1}^{n}\big((p+1)\cdot p!-p\cdot(p-1)!\big)\\
 & =\sum_{p=1}^{n}(a_{p}-a_{p-1}) & \text{en posant, pour tout }p\in\mathbb{N},\ a_{p}=(p+1)\cdot p!\\
 & =-\sum_{k=0}^{n-1}(a_{k}-a_{k+1}) & \text{par le changement d'indice }k=p-1\\
 & =-(a_{0}-a_{n}) & \text{d'apr�s la formule de t�lescopage}\\
 & =(n+1)\cdot n!-1\\
 & =(n+1)!-1
\end{aligned}
\]
}{\large \par}

\selectlanguage{french}%
\bordure\selectlanguage{english}%

\end{document}
