%% LyX 2.2.1 created this file.  For more info, see http://www.lyx.org/.
%% Do not edit unless you really know what you are doing.
\documentclass[french,12pt]{paper}
\usepackage[T1]{fontenc}
\usepackage[utf8]{inputenc}
\usepackage[landscape,a4paper]{geometry}
\geometry{verbose,tmargin=27mm,bmargin=20mm,lmargin=40mm,rmargin=40mm}
\pagestyle{empty}
\setlength{\parskip}{\bigskipamount}
\setlength{\parindent}{0pt}
\usepackage{amsmath}
\usepackage{amssymb}

\makeatletter
%%%%%%%%%%%%%%%%%%%%%%%%%%%%%% User specified LaTeX commands.
\usepackage{tikz}
\usetikzlibrary{calc}
\usetikzlibrary{shapes}
\usepackage{mathrsfs}
\usepackage{mathabx}
\usepackage{txfonts}
\usepackage{pxfonts}
\usepackage{titling}
\usepackage{array}
%\usepackage{yhmath}

\newdimen\un \un=1mm
\def\bordure{
\begin{tikzpicture}[overlay,remember picture]
\node(Triskell) at ($(current page.north west)+(21*\un,-20*\un)$){};
\def\p{.75}
\def\ang{45}
\def\alp{160.2}
\def\bet{72.42}
\def\gam{-13.2}
\draw [very thick, line width = 8pt, color = red!25!blue!33.333!green!50]
($(current page.north west)+(20*\un,-20*\un)$)
-- ($(current page.north east)+(-24*\un,-20*\un)$)
node [draw, ellipse, fill, text=white, pos=.5] {\thetitle}
arc (90 : 0 : 4*\un)
-- ($(current page.south east)+(-20*\un,24*\un)$)
node[text=white,pos=.5 , rotate=90]
{\tiny Ce document est sous licence GNU FDL,
 il est librement modifiable et distribuable.
 Sources et licence complètes disponible sur le site.
 Copyright 2012, Jean-Christophe Jameux}
arc (0 : -90 : 4*\un)
-- ($(current page.south west)+(24*\un,20*\un)$)
node[draw, ellipse, fill, text=white, pos=.15] {\bf Echologie.org}
arc (-90 : -180 : 4*\un)
-- ($(current page.north west)+(20*\un,-20*\un)$);
\draw [fill = white, color = red!25!blue!33.333!green!50]
(Triskell) + (1.2*\un,-6*\un) circle (15*\un);
\draw [fill = white, color = white] (Triskell) circle (2*\un);
\draw [fill = white, color = white]
(Triskell) + ({120*(1+\p)} : 3*\un)
arc ({120*(1+\p)} : 120 : 3*\un)
arc (180+\ang : 180-\ang :3*\un)
arc (\alp-\ang : \alp+\ang+24.8 : 5*\un);
\draw [fill = white, color = white]
(Triskell) + (120*\p : 3*\un)
arc (120*\p : 0 : 3*\un)
arc (90+\ang : 90-\ang : 6*\un)
arc (\bet-\ang : \bet+\ang+5.9 : 8*\un);
\draw [fill = white, color = white]
(Triskell) + ({120*(2+\p)} : 3*\un)
arc ({120*(2+\p)} : 240 : 3*\un)
arc (\ang : -\ang : 12*\un)
arc (\gam-\ang : \gam+\ang+.85 : 13*\un);
\end{tikzpicture}}

\makeatother

\usepackage{babel}
\makeatletter
\addto\extrasfrench{%
   \providecommand{\og}{\leavevmode\flqq~}%
   \providecommand{\fg}{\ifdim\lastskip>\z@\unskip\fi~\frqq}%
}

\makeatother
\begin{document}
\title{\bf Vrai/Faux sur les quadrilatères}

Un quadrilatère symétrique par rapport aux médiatrices des côtés est
un carré. \vfill{}
Un quadrilatère ayant deux côtés opposés parallèles est un parallélogramme.
\vfill{}
Un quadrilatère ayant ses côtés opposés deux à deux de même longueur
est un parallélogramme. \vfill{}
Un quadrilatère ayant ses diagonales de même longueur est un rectangle.
\vfill{}
Un quadrilatère ayant quatre côtés de même longueur est un carré.
\vfill{}
Un quadrilatère dont les diagonales se coupent en leur milieux est
un parallélogramme. \vfill{}
Un parallélogramme ayant deux côtés consécutifs de même longueur est
un losange. \vfill{}
Un quadrilatère symétrique par rapport à ses diagonales est un carré.
\vfill{}
Un quadrilatère ayant trois angles droits est un rectangle. \vfill{}
Un parallélogramme ayant un angle droit est un rectangle. \vfill{}
Un quadrilatère dont les angles sont deux à deux de même mesure est
un parallélogramme. \vfill{}
Un quadrilatère dont les angles opposés sont deux à deux de même mesure
est un parallélogramme. \vfill{}
Un quadrilatère dans lequel il existe deux angles consécutifs supplémentaires
est un parallélogramme. \vfill{}
Un quadrilatère dont les sommets sont symétriques par rapport a l'intersection
des diagonales est un losange. \vfill{}
Un parallélogramme dont les diagonales se coupent à angle droit est
un losange. \vfill{}
Un parallélogramme ayant deux côtés consécutifs de même longueur est
un losange. \vfill{}
Un quadrilatère symétrique par rapport aux médiatrices des côtés est
un carré. \vfill{}
Un quadrilatère inscriptible dans un cercle est un rectangle. \vfill{}
Un quadrilatère ayant deux côtés opposés parallèles est un parallélogramme.
\vfill{}
Un quadrilatère ayant ses côtés opposés deux à deux de même longueur
est un parallélogramme.

\vfill{}

\bordure
\end{document}
