%% LyX 2.2.1 created this file.  For more info, see http://www.lyx.org/.
%% Do not edit unless you really know what you are doing.
\documentclass[english,12pt]{paper}
\usepackage[T1]{fontenc}
\usepackage[latin9]{inputenc}
\usepackage[landscape,a4paper]{geometry}
\geometry{verbose,tmargin=30mm,bmargin=20mm,lmargin=35mm,rmargin=35mm}
\pagestyle{empty}
\setlength{\parskip}{\bigskipamount}
\setlength{\parindent}{0pt}
\usepackage{amsmath}
\usepackage{amssymb}

\makeatletter
%%%%%%%%%%%%%%%%%%%%%%%%%%%%%% User specified LaTeX commands.
\usepackage{tikz}
\usetikzlibrary{calc}
\usetikzlibrary{shapes}
\usepackage{mathrsfs}
\usepackage{mathabx}
\usepackage{txfonts}
\usepackage{pxfonts}
\usepackage{titling}
\usepackage{array}

\newdimen\un \un=1mm
\def\bordure{
\begin{tikzpicture}[overlay,remember picture]
\node(Triskell) at ($(current page.north west)+(21*\un,-20*\un)$){};
\def\p{.75}
\def\ang{45}
\def\alp{160.2}
\def\bet{72.42}
\def\gam{-13.2}
\draw [very thick, line width = 8pt, color = red!25!blue!33.333!green!50]
($(current page.north west)+(20*\un,-20*\un)$)
-- ($(current page.north east)+(-24*\un,-20*\un)$)
node [draw, ellipse, fill, text=white, pos=.5] {\thetitle}
arc (90 : 0 : 4*\un)
-- ($(current page.south east)+(-20*\un,24*\un)$)
node[text=white,pos=.5 , rotate=90]
{\tiny Ce document est sous licence GNU FDL,
 il est librement modifiable et distribuable.
 Sources et licence compl�tes disponible sur le site.
 Copyright 2016, Jean-Christophe Jameux}
arc (0 : -90 : 4*\un)
-- ($(current page.south west)+(24*\un,20*\un)$)
node[draw, ellipse, fill, text=white, pos=.15] {\bf Echologie.org}
arc (-90 : -180 : 4*\un)
-- ($(current page.north west)+(20*\un,-20*\un)$);
\draw [fill = white, color = red!25!blue!33.333!green!50]
(Triskell) + (1.2*\un,-6*\un) circle (15*\un);
\draw [fill = white, color = white] (Triskell) circle (2*\un);
\draw [fill = white, color = white]
(Triskell) + ({120*(1+\p)} : 3*\un)
arc ({120*(1+\p)} : 120 : 3*\un)
arc (180+\ang : 180-\ang :3*\un)
arc (\alp-\ang : \alp+\ang+24.8 : 5*\un);
\draw [fill = white, color = white]
(Triskell) + (120*\p : 3*\un)
arc (120*\p : 0 : 3*\un)
arc (90+\ang : 90-\ang : 6*\un)
arc (\bet-\ang : \bet+\ang+5.9 : 8*\un);
\draw [fill = white, color = white]
(Triskell) + ({120*(2+\p)} : 3*\un)
arc ({120*(2+\p)} : 240 : 3*\un)
arc (\ang : -\ang : 12*\un)
arc (\gam-\ang : \gam+\ang+.85 : 13*\un);
\end{tikzpicture}}

\makeatother

\usepackage{babel}
\begin{document}
\title{Mise en �quations}

\textbf{\large{}\hphantom{\textbf{\large{}D}}Dans la bo�te � outil
du math�maticien, la mise en �quation est l'arme la plus destructrice...
De difficult�s ! Donnons-en un exemple : }{\large \par}
\begin{center}
\textbf{\large{}}%
\begin{minipage}[t]{0.8\columnwidth}%
\begin{center}
\textbf{\large{}Un nombre entier a pour reste 35 dans la division
euclidienne par 69, et dans la division par 75, il a m�me quotient
et pour reste 17. Quel est ce nombre ?}
\par\end{center}%
\end{minipage}
\par\end{center}{\large \par}

\textbf{\large{}On peut tout � fait trouver la solution de ce probl�me
en tatonant, mais rien ne garantit qu'on aboutira, et pire, face �
un autre probl�me, on repart de z�ro. La clef de la mise en �quation,
c'est de nomer les �l�ments du probl�me, afin de pouvoir formuler
le probl�me comme si l'on connaissait la solution. Dans notre exemple,
on peut appeler $n$ le nombre entier recherch�, et $q$ le quotient
commun. La r�solution du probl�me arrive tr�s vite apr�s la mise en
�quation :
\begin{align*}
\begin{cases}
n=69q+35\\
n=75q+17
\end{cases} & \Longrightarrow0=-6q+18 & \textrm{En soustrayant les deux lignes}\\
 & \Longrightarrow q=3 & \textrm{On trouve ainsi le quotient commun}\\
 & \Longrightarrow n=69\times3+35 & \textrm{On substitue dans la premi�re �quation}\\
 & \Longrightarrow n=242 & \textrm{(ou la deuxi�me : \ensuremath{n=75\times3+17=242})}
\end{align*}
En faisant ainsi, on ne d�montre pas que 242 est une solution de notre
probl�me, puisque nous avons des implications et non des �quivalences.
Ce que nous avons montr�, c'est que seul 242 peut �tre solution, mais
c'est un jeu d'enfant de v�rifier que ce nombre r�pond bien aux conditions
du probl�me. On a ainsi fait mieux que trouver une solution du probl�me,
on a prouv� qu'il n'y en avait pas d'autres, et on a surtout, en prenant
un peu de recul, mis en pratique une technique de r�solution de probl�me
tr�s puissante et tr�s g�n�rale.}{\large \par}

\bordure
\end{document}
