\documentclass[a4paper, landscape, 10pt]{article}
\usepackage[top=20mm,left=24mm,right=24mm,bottom=15mm]{geometry}
\usepackage[T1]{fontenc}
\usepackage[utf8]{inputenc}
\usepackage[francais]{babel}
\usepackage{tikz}
\usetikzlibrary{calc}
\usetikzlibrary{shapes}
\usepackage{amsmath}
\usepackage{amssymb}
\usepackage{mathrsfs}
\usepackage{mathabx}
%\usepackage{ulsy}
\usepackage{txfonts}
\usepackage{pxfonts}

\begin{document}
\pagestyle{empty}
\parskip1mm
\parindent0mm
\centering
\def\titre{\LARGE Propriétés des quadrilatères}

Les propositions en gras sont des définitions.

Deux propositions faces à faces sont réciproques l'une de l'autre.

A gauche se trouvent les propriétés des quadrilatères particuliers. On les utilise quand on sait qu'un quadrilatère est un parallélogramme\\
et qu'on veut en déduire des propriétés sur la figure.

Leurs réciproques, à droite, sont les théorèmes permettant de démontrer qu'un quadrilatère particulier est\\
un parallélogramme, un rectangle, un losange, ou encore un carré.

\vfill

\catcode`|13
\long\def|#1|#2|{
    \vskip1.5mm
    \hbox to\hsize{
        \vtop{
            \raggedleft
            \hsize108mm#1
        }
        \hfill
        \vtop{
            \raggedright
            \hsize128mm#2
        }
    }
}

|Dans un parallélogramme, les côtés opposés\\ sont deux à deux parallèles.|{\bf Un quadrilatère ayant ses côtés opposés deux à deux\\ parallèles est un parallélogramme.}|

|Dans un parallélogramme, les côtés opposés\\ sont deux à deux de même longueur.|Un quadrilatère ayant ses côtés opposés deux à deux\\ de même longueur est un parallélogramme.|

|Les diagonales d'un parallélogramme\\se coupent en leur milieux.|Un quadrilatère dont les diagonales se coupent\\ en leur milieux est un parallélogramme.|

|Dans un parallélogramme, deux angles consécutifs\\sont toujours supplémentaires.|Un quadrilatère dans lequel deux angles consécutifs\\ sont toujours supplémentaires est un parallélogramme.|

|Dans un parallélogramme, les angles opposés\\ sont deux à deux de même mesure.|Un quadrilatère dont les angles opposés sont deux\\ à deux de même mesure est un parallélogramme.|

|Les parallélogrammes sont symétriques par\\ rapport au centre de leurs diagonales.|Un quadrilatère dont les sommets sont symétriques\\ par rapport au centre de ses diagonales est un parallélogramme.|

|Un rectangle a quatre angles droits.|{\bf Un quadrilatère ayant quatre angles droits est un rectangle.} (trois suffisent)|

|Tous les angles d'un rectangle sont droits.|Un {\it parallélogramme} ayant un angle droit est un rectangle.|

|Un rectangle a ses diagonales de même longueur.|Un {\it parallélogramme} ayant ses diagonales de même longueur est un rectangle.|

|Un rectangle est symétrique par rapport\\ aux médiatrices de ses côtés.|Un quadrilatère symétrique par rapport aux\\ médiatrices de ses côtés est un rectangle.|

|Un rectangle est inscriptible dans un cercle.|Un {\it parallélogramme} inscriptible dans un cercle est un rectangle.|

|Un losange a ses quatre côtés de même longueur.|{\bf Un quadrilatère ayant quatre côtés de même longueur est un losange.}|

|Deux côtés consécutifs d'un losange\\ sont toujours de même longueur.|Un {\it parallélogramme} ayant deux côtés consécutifs\\ de même longueur est un losange.|

|Les diagonales d'un losange se coupent à angle droit.|Un {\it parallélogramme} dont les diagonales se coupent à angle droit est un losange.|

|Les losanges sont symétriques par rapport\\ aux axes portés par leurs diagonales.|Un quadrilatère symétrique par rapport aux axes\\ portés par ses diagonales est un losange.|

|Les rectangles et les losanges sont des parallélogrammes.\\Ils héritent donc de toutes leurs propriétés.|Un quadrilatère qui est un rectangle ou un losange\\ est aussi un parallélogramme.|

\vfill

On rappelle qu'un carré est un quadrilatère qui est à la fois un rectangle et un losange. Il a donc l'intégralité des propriétés énoncées sur cette page.\\
Si l'on souhaite démontrer qu'un quadrilatère est un carré, il faut démontrer à la fois que c'est un rectangle et un losange.

\newdimen\un
\un=1.2mm
\newdimen\deux
\deux=.45mm

\begin{tikzpicture}[overlay,remember picture]

\def\p{.75}
\def\ang{45}
\def\alp{160.2}
\def\bet{72.42}
\def\gam{-13.2}

\node(Triskell) at ($(current page.north west)+(15*\un,-10*\un)$){};

\draw[very thick, line width = 10pt, color=red!25!blue!33.333!green!50]
    ($(current page.north west)+(20*\deux,-20*\deux)$) --
	($(current page.north east)+(-24*\deux,-20*\deux)$)
	node [draw, ellipse, fill, text=white, pos=.5]
	{\Large\titre} arc 
	(90 : 0 : 4*\deux) -- ($(current page.south east)+(-20*\deux,24*\deux)$)
	node[text=white,pos=.5 , rotate=90]
		{\scriptsize Ce document est sous licence GNU FDL,
		 il est librement modifiable et distribuable.
		 Sources et licence complètes disponible sur le site.
		 Copyright 2012, Jean-Christophe Jameux}
	arc (0 : -90 : 4*\deux) -- ($(current page.south west)+(24*\deux,20*\deux)$)
	node[draw, ellipse, fill, text=white, pos=.15]
	{\Large\bf Echologie.org} arc 
	(-90 : -180 : 4*\deux) -- ($(current page.north west)+(20*\deux,-20*\deux)$);
\draw[fill=white,color=red!25!blue!33.333!green!50]
    (Triskell) + (1.2*\un,-6*\un) circle (15*\un);
\draw[fill=white,color=white]	(Triskell) circle (2*\un);
\draw[fill=white,color=white]
    (Triskell) + ({120*(1+\p)} : 3*\un) arc
				 ({120*(1+\p)} : 120 : 3*\un) arc
				 (180+\ang : 180-\ang :3*\un) arc
				 (\alp-\ang : \alp+\ang+24.8 : 5*\un);
\draw[fill=white,color=white]
    (Triskell) + (120*\p : 3*\un) arc
				 (120*\p : 0 : 3*\un) arc
				 (90+\ang : 90-\ang : 6*\un) arc
				 (\bet-\ang : \bet+\ang+5.9 : 8*\un);
\draw[fill=white,color=white]	(Triskell)+
				({120*(2+\p)} : 3*\un) arc
				({120*(2+\p)} : 240 : 3*\un) arc
				(\ang : -\ang : 12*\un) arc
				(\gam-\ang : \gam+\ang+.85 : 13*\un);

\end{tikzpicture}

\end{document}
